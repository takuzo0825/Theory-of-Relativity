%%% luaLaTeX
\usepackage{luatexja}% LuaLaTexの日本語化
\usepackage{luatexja-fontspec} 
\usepackage[haranoaji,nfssonly]{luatexja-preset} % 原ノ味フォント
%%% 目次
%目次でどこまで表示するか 3:subsubsectionまで 2:subsectionまで
\setcounter{tocdepth}{2}
\usepackage[shortlabels]{enumitem}
\newcommand\boldred[1]{\textcolor{red!60!black}{\textbf{#1}}}
%%% 数式
\usepackage{amsmath,amssymb}
\usepackage{mathtools}
\usepackage{empheq} % 連立方程式
\numberwithin{equation}{section}
\DeclareMathOperator{\id}{id}
%%% 定理
\usepackage{tcolorbox}
    \tcbuselibrary{breakable, skins, theorems}
    \tcbset{enhanced,fonttitle=\bfseries,boxrule=0pt,frame hidden,
        detach title, before upper={\tcbtitle\par},
        separator sign none,colback=white,
        description delimiters parenthesis}
    \newtcbtheorem[number within=chapter,list inside={Dfn}]{dfn}{定義}{
        coltitle=red!60!black,colback=red!5!white
    }{dfn}
    \newtcbtheorem[use counter from=dfn,list inside={Thm}]{lem}{補題}{
        coltitle=blue!60!black,colback=blue!5!white
    }{lem}
    \newtcbtheorem[use counter from=dfn,list inside={Thm}]{prop}{命題}{
        coltitle=blue!60!black,colback=blue!5!white
    }{prop}
    \newcommand{\propref}[1]{命題~\ref{prop:#1}}
    \newtcbtheorem[use counter from=dfn,list inside={Thm}]{thm}{定理}{
        coltitle=blue!60!black,colback=blue!5!white
    }{thm}
    \newcommand{\thmref}[1]{定理~\ref{thm:#1}}
    \newtcbtheorem{exm}{例}{
        coltitle=black,borderline west = {1pt}{0pt}{black}
    }{exm}
    \newtcbtheorem{prf}{証明}{
        coltitle=black,colback=black!5!white,after upper={\hfill$\Box$}
    }{proof}
%%% 物理
\usepackage{physics}
\usepackage{siunitx}
\sisetup{per-mode = symbol}
%%% 化学
\usepackage[version=3]{mhchem}
%%% 画像
\usepackage{graphicx}
\usepackage{subcaption}
\renewcommand{\figurename}{図}
%%% ハイパーリンク
\usepackage{hyperref}
%%% 参照
\usepackage{cleveref}
\newcommand{\crefrangeconjunction}{--} % 3つ以上連番で参照するときの表記を設定
\crefname{equation}{式}{式}
\Crefname{equation}{式}{式}
\crefname{dfn}{定義}{定義}
\Crefname{dfn}{定義}{定義}
\crefname{lem}{補題}{補題}
\crefname{figure}{図}{図}
\crefname{table}{表}{表}
\crefname{chapter}{第}{第}
\creflabelformat{chapter}{#2#1章#3}
\crefname{section}{第}{第}
\creflabelformat{section}{#2#1節#3}
\crefname{subsection}{第}{第}
\creflabelformat{subsection}{#2#1項#3}
\crefname{subsubsection}{第}{第}
\creflabelformat{subsubsection}{#2#1目#3}
\newcommand{\crefpairconjunction}{と}
\renewcommand{\crefrangeconjunction}{から}
\newcommand{\crefmiddleconjunction}{,}
\newcommand{\creflastconjunction}{,および}
%%% 参考文献
% bibtex
\usepackage{url}
\usepackage[backend = biber,style = phys]{biblatex}
\ExecuteBibliographyOptions{%
    sorting = none, % 引用リストの表示順
    sortcites = true, % 本文での引用順番
    maxnames = 8, %
}
\bibliography{ref.bib}
%%% ファイル分割
\usepackage{subfiles}
%%%