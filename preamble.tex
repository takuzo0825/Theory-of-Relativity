%%% luaLaTeX
\usepackage{luatexja}
% \usepackage{luatexja-fontspec}
\usepackage[no-math,deluxe,expert,haranoaji]{luatexja-preset}
%%% Table of Contents
\setcounter{tocdepth}{2}
%%% Itemization
\usepackage[shortlabels]{enumitem}
%%% Figure
\renewcommand{\figurename}{図}
\usepackage{xcolor}
\usepackage{caption}
\captionsetup{format=hang,justification=raggedright}
\usepackage{subcaption}
\captionsetup[sub]{
    skip=0pt,subrefformat=simple,labelformat=simple,
    justification=raggedright,singlelinecheck=false
}
\renewcommand\thesubfigure{\,(\alph{subfigure})}
\usepackage{tikz}
\usepackage{tcolorbox}
\tcbuselibrary{breakable, skins, theorems}
\tcbset{enhanced,fonttitle=\bfseries,boxrule=0pt,frame hidden,
    detach title, before upper={\tcbtitle\par},
    separator sign none,colback=white,
    description delimiters parenthesis}
\newtcbtheorem[number within=chapter,list inside={Dfn}]{dfn}{定義}{
    coltitle=red!60!black,colback=red!5!white
}{dfn}
\newtcbtheorem[use counter from=dfn,list inside={Thm}]{lem}{補題}{
    coltitle=blue!60!black,colback=blue!5!white
}{lem}
\newtcbtheorem[use counter from=dfn,list inside={Thm}]{prop}{命題}{
    coltitle=blue!60!black,colback=blue!5!white
}{prop}
\newcommand{\propref}[1]{命題~\ref{prop:#1}}
\newtcbtheorem[use counter from=dfn,list inside={Thm}]{thm}{定理}{
    coltitle=blue!60!black,colback=blue!5!white
}{thm}
\newcommand{\thmref}[1]{定理~\ref{thm:#1}}
\newtcbtheorem{exm}{例}{
    breakable,coltitle=black,borderline west = {1pt}{0pt}{black}
}{exm}
\newtcbtheorem{prf}{証明}{
    breakable,coltitle=black,colback=black!5!white,after upper={\hfill$\Box$}
}{proof}
%%% Link
\usepackage{url}
\usepackage{hyperref}
\hypersetup{
	colorlinks=false, % リンクに色をつけない設定
	bookmarks=true, % 以下ブックマークに関する設定
	bookmarksnumbered=true,
	pdfborder={0 0 0},
	bookmarkstype=toc
}
%%% Reference
\usepackage{cleveref}
\newcommand{\crefrangeconjunction}{--} % 3つ以上連番で参照するときの表記を設定
\crefname{equation}{式}{式}
\Crefname{equation}{式}{式}
\crefname{dfn}{定義}{定義}
\Crefname{dfn}{定義}{定義}
\crefname{lem}{補題}{補題}
\crefname{figure}{図}{図}
\crefname{table}{表}{表}
\crefname{chapter}{第}{第}
\creflabelformat{chapter}{#2#1章#3}
\crefname{section}{第}{第}
\creflabelformat{section}{#2#1節#3}
\crefname{subsection}{第}{第}
\creflabelformat{subsection}{#2#1項#3}
\crefname{subsubsection}{第}{第}
\creflabelformat{subsubsection}{#2#1目#3}
\newcommand{\crefpairconjunction}{と}
\renewcommand{\crefrangeconjunction}{から}
\newcommand{\crefmiddleconjunction}{,}
\newcommand{\creflastconjunction}{,および}
\usepackage[backend = biber,style = phys]{biblatex}
\ExecuteBibliographyOptions{%
    sorting = none, % 引用リストの表示順
    sortcites = true, % 本文での引用順番
    maxnames = 8, %
}
\bibliography{ref.bib}
%%% mathmatics
\numberwithin{equation}{section}
\usepackage{amsmath}
\usepackage{amssymb}
\usepackage{mathtools}
\usepackage{empheq}
%%% Physics
\usepackage{physics}
\usepackage{tensor}
\usepackage{siunitx}
\sisetup{per-mode = symbol}
%%% Chemical
\usepackage[version=3]{mhchem}
%%% File
\usepackage{subfiles}
%%% User Define
\newcommand\boldred[1]{\textcolor{red!60!black}{\textbf{#1}}}
\DeclareMathOperator{\id}{id}
\DeclareMathOperator{\sgn}{sgn}
\DeclareMathOperator{\supp}{supp}